\documentclass[a4paper,UKenglish]{lipics}
%This is a template for producing LIPIcs articles. 
%See lipics-manual.pdf for further information.
%for A4 paper format use option "a4paper", for US-letter use option "letterpaper"
%for british hyphenation rules use option "UKenglish", for american hyphenation rules use option "USenglish"
% for section-numbered lemmas etc., use "numberwithinsect"
 
\usepackage{microtype}%if unwanted, comment out or use option "draft"

%\graphicspath{{./graphics/}}%helpful if your graphic files are in another directory

\bibliographystyle{plain}% the recommended bibstyle

% Author macros::begin %%%%%%%%%%%%%%%%%%%%%%%%%%%%%%%%%%%%%%%%%%%%%%%%
\title{Exploring Circle Packing Algorithms\footnote{This work was partially supported by the National Science Foundation under grants CCF-1525978 and CCF-1464379.}}
\titlerunning{Exploring Circle Packing Algorithms} %optional, in case that the title is too long; the running title should fit into the top page column
 
\author[1]{Kevin Pratt}
\author[2]{Connor Riley}
\author[3]{Donald R.~Sheehy}
\affil[1]{University of Connecticut\\
  \texttt{kevin.pratt@uconn.edu}}
\affil[2]{University of Connecticut\\
  \texttt{connor.riley@uconn.edu}}
\affil[3]{University of Connecticut\\
  \texttt{don.r.sheehy@gmail.com}}

\authorrunning{K.\,Pratt, C.\,Riley, and D.\,R.\,Sheehy} %mandatory. First: Use abbreviated first/middle names. Second (only in severe cases): Use first author plus 'et. al.'

\Copyright{Kevin Pratt, Connor Riley, Donald R.~Sheehy}%mandatory, please use full first names. LIPIcs license is "CC-BY";  http://creativecommons.org/licenses/by/3.0/

\subjclass{F.2.2 Nonnumerical Algorithms and Problems---Geometrical problems and computations.}
% \subjclass{F.2.2 Nonnumerical Algorithms and Problems}% mandatory: Please choose ACM 1998 classifications from http://www.acm.org/about/class/ccs98-html . E.g., cite as "F.1.1 Models of Computation".
\keywords{Computational Geometry, Processing, Javascript, Visualisation, Incremental Algorithms}% mandatory: Please provide 1-5 keywords

% Author macros::end %%%%%%%%%%%%%%%%%%%%%%%%%%%%%%%%%%%%%%%%%%%%%%%%%

%Editor-only macros:: begin (do not touch as author)%%%%%%%%%%%%%%%%%%%%%%%%%%%%%%%%%%
\serieslogo{}%please provide filename (without suffix)
\volumeinfo%(easychair interface)
  {Billy Editor and Bill Editors}% editors
  {2}% number of editors: 1, 2, ....
  {Conference title on which this volume is based on}% event
  {1}% volume
  {1}% issue
  {1}% starting page number
\EventShortName{}
\DOI{10.4230/LIPIcs.xxx.yyy.p}% to be completed by the volume editor
% Editor-only macros::end %%%%%%%%%%%%%%%%%%%%%%%%%%%%%%%%%%%%%%%%%%%%%%%

\begin{document}

\maketitle

\begin{abstract}
  We present an online, interactive tool for visualizing and experimenting with different circle packing algorithms.
\end{abstract}

\section{Introduction} % (fold)
\label{sec:introduction}

  The Koebe Embedding Theorem provides a wonderful link between questions of geometry, topology, combinatorics, and complex analysis.
  It states that every planar graph can be realized as the intersection graph of a collection of interior disjoint circles in the plane.
  That is, the vertices correspond to circles and two circles are tangent if and only if their corresponding vertices share an edge.
  
  % Although originally proven by Koebe in the 1920's it's history was 

% section introduction (end)

\section{Background} % (fold)
\label{sec:background}

  Circle Packings
  Tutte's Algorithm
  Maxwell's Theorem
  Weighted Delaunay
  Dual Packings

% section background (end)Background

\section{Interactive Input} % (fold)
\label{sec:interactive_input}

  One immediate challenge to working interactively with circle packings is the need for input triangulations.
  We would like to be able to produce triangulations with a minimum of effort.  
  The most natural approach is to draw circles and produce the weighted Delaunay triangulation of the circles.
  This is the dual to the so-called power diagram, a Voronoi diagram on the points where the (squared) distance to a circle $c$ with center $p$ and radius $r$ is defined to be
  \[
    \pi_c(x)^2 := \|x-p\|^2 - r^2.
  \]
  
  Weighted Delaunay triangulations have another interpretation as projections of convex polyhedra in $\R^3$

  Drop weighted points and build the weighted Delaunay triangulation incrementally
  (advantage: you can draw an approximate packing)
  
  Mobius Transformations
  Stereographic Maps?
  Parabolic Lifting?
  
% section interactive_input (end)

\section{Algorithms} % (fold)
\label{sec:algorithms}

  Springs
  
  Dual Springs
  
  Stephenson Algorithm
  
  All are continuous embeddings

% section algorithms (end)

\section{Future Work} % (fold)
\label{sec:future_work}

  Incremental Algorithms
  Curvature Flows

% section future_work (end)
%%
%% Bibliography
%%

%% Either use bibtex (recommended), but commented out in this sample

%\bibliography{dummybib}

%% .. or use bibitems explicitely

\nocite{Simpson}

\begin{thebibliography}{50}
\bibitem{Simpson} Homer J. Simpson. \textsl{Mmmmm...donuts}. Evergreen Terrace Printing Co., Springfield, Somewhere, USA, 1998
\end{thebibliography}


\end{document}
